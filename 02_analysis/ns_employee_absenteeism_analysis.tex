% Options for packages loaded elsewhere
\PassOptionsToPackage{unicode}{hyperref}
\PassOptionsToPackage{hyphens}{url}
%
\documentclass[
]{article}
\usepackage{lmodern}
\usepackage{amssymb,amsmath}
\usepackage{ifxetex,ifluatex}
\ifnum 0\ifxetex 1\fi\ifluatex 1\fi=0 % if pdftex
  \usepackage[T1]{fontenc}
  \usepackage[utf8]{inputenc}
  \usepackage{textcomp} % provide euro and other symbols
\else % if luatex or xetex
  \usepackage{unicode-math}
  \defaultfontfeatures{Scale=MatchLowercase}
  \defaultfontfeatures[\rmfamily]{Ligatures=TeX,Scale=1}
\fi
% Use upquote if available, for straight quotes in verbatim environments
\IfFileExists{upquote.sty}{\usepackage{upquote}}{}
\IfFileExists{microtype.sty}{% use microtype if available
  \usepackage[]{microtype}
  \UseMicrotypeSet[protrusion]{basicmath} % disable protrusion for tt fonts
}{}
\makeatletter
\@ifundefined{KOMAClassName}{% if non-KOMA class
  \IfFileExists{parskip.sty}{%
    \usepackage{parskip}
  }{% else
    \setlength{\parindent}{0pt}
    \setlength{\parskip}{6pt plus 2pt minus 1pt}}
}{% if KOMA class
  \KOMAoptions{parskip=half}}
\makeatother
\usepackage{xcolor}
\IfFileExists{xurl.sty}{\usepackage{xurl}}{} % add URL line breaks if available
\IfFileExists{bookmark.sty}{\usepackage{bookmark}}{\usepackage{hyperref}}
\hypersetup{
  pdftitle={Nova Scotia Employee Absenteeism - in Python},
  pdfauthor={Chris Selig},
  hidelinks,
  pdfcreator={LaTeX via pandoc}}
\urlstyle{same} % disable monospaced font for URLs
\usepackage[margin=1in]{geometry}
\usepackage{graphicx,grffile}
\makeatletter
\def\maxwidth{\ifdim\Gin@nat@width>\linewidth\linewidth\else\Gin@nat@width\fi}
\def\maxheight{\ifdim\Gin@nat@height>\textheight\textheight\else\Gin@nat@height\fi}
\makeatother
% Scale images if necessary, so that they will not overflow the page
% margins by default, and it is still possible to overwrite the defaults
% using explicit options in \includegraphics[width, height, ...]{}
\setkeys{Gin}{width=\maxwidth,height=\maxheight,keepaspectratio}
% Set default figure placement to htbp
\makeatletter
\def\fps@figure{htbp}
\makeatother
\setlength{\emergencystretch}{3em} % prevent overfull lines
\providecommand{\tightlist}{%
  \setlength{\itemsep}{0pt}\setlength{\parskip}{0pt}}
\setcounter{secnumdepth}{-\maxdimen} % remove section numbering

\title{Nova Scotia Employee Absenteeism - in Python}
\author{Chris Selig}
\date{08 August, 2020}

\begin{document}
\maketitle

\hypertarget{summary}{%
\section{Summary}\label{summary}}

Way back in August 2017 I started to apply R to working with data and
one of my early projects was to explore the Nova Scotia (NS) Government
employee absentee
\href{https://data.novascotia.ca/Public-Service/Nova-Scotia-Government-Employee-Absenteeism/3kpf-veux}{data.}
You can see the original post
\href{https://bidamia.ca/post/nova-scotia-government-employee-absentee-analysis/}{here.}

I am revisiting this analysis because I am now learning python and I got
tired of doing courses and wanted to apply what I learned so far. Also,
there is now a few years more worth of data.

Goal:

The code for this analysis can be found on my
\href{https://github.com/chrisselig/ns_employee_absenteeism_python}{Github
page.}

\hypertarget{the-data}{%
\section{The Data}\label{the-data}}

As before, I will start with an exploratory analysis of the data, but
first, let's review the columns in the dataset.

\begin{enumerate}
\def\labelenumi{\arabic{enumi}.}
\item
  absence\_date: The date of absence for the employee
\item
  absence\_hours: The number of hours the employee was absent
\item
  absence\_type: The reason for the absence
\item
  absence\_type\_category: The absence reasons are grouped into standard
  categories
\item
  age\_cohort\_on\_absence\_date: The age range of the employee, the day
  they were absent
\item
  employee\_type: Whether the employee was union, or non-union
\item
  gender: The gender of the employee
\end{enumerate}

In total, there are 7 columns and 3,568,910 rows with data spanning from
April 1st 2014 to March 31st, 2020.

'

\n  

\n    

\n      

absence\_date

\n      

absence\_type\_category

\n      

absence\_type

\n      

absence\_hours

\n      

employee\_type

\n    

\n  

\n  

\n    

\n      

2014-04-01

\n      

Short Term Illness

\n      

Short Term Illness

\n      

8.00

\n      

Union

\n    

\n    

\n      

2014-04-01

\n      

Short Term Illness

\n      

Short Term Illness

\n      

9.00

\n      

Union

\n    

\n    

\n      

2014-04-01

\n      

Short Term Illness

\n      

Short Term Illness

\n      

9.00

\n      

Union

\n    

\n    

\n      

2014-04-01

\n      

General Illness

\n      

General Illness

\n      

7.75

\n      

Union

\n    

\n  

\n

'

\hypertarget{exploratory-analysis}{%
\section{Exploratory Analysis}\label{exploratory-analysis}}

Originally, I started by taking a look at the total number of days
absent which not only did not account for the fact that the people were
not always absent for full days but I also only looked at total
absences. A more realistic approach would have been to standardize
absences per 100 people or absence hours per 100 people. Unfortunately,
I cannot find any information on how many people are employed by the NS
government so I will continue with counts alone.

Below is the count of absences by age group. Interestingy enough, the
initial trend did not hold. When I looked at this a few years ago, the
50-54 age category had the most absences. Now it is the 55-59 crowd. I
wonder if that is because the 50 - 54 crowd could now be in the 55 - 59
crowd?

At the bottom of the scale, the under 25 crowd still has the lowest
absences. Either way, I do not want to infer much from this data because
I have no idea of the counts of each employee cohort.

\includegraphics{ns_employee_absenteeism_analysis_files/figure-latex/unnamed-chunk-7-1.pdf}

Moving on, let's take a look at the counts of absences by category.

\includegraphics{ns_employee_absenteeism_analysis_files/figure-latex/unnamed-chunk-8-1.pdf}

The first thing I noticed about this plot was that there is a new
category: Pandemic\_Leave. At the time of writing (August 2020), the
world is still experiencing COVID-19 and alot of places in Canada are
still working from home, evening though restrictions have been loosening
across the country.

Family illnesses seems to be the most popular reason for an absence
across all the age categories. As before, short and long term illnesses
increase as the employee age increases, which makes total sense because
people tend to develop health issues as they get older.

Next, I will explore absences by month.

\includegraphics{ns_employee_absenteeism_analysis_files/figure-latex/countByMonth-1.pdf}

As before, there is a spike in March, July, August and December. March
spiked probaby relate to it being March break for schools. Parents often
take time off for vacationing with their children at this time. July and
August are the most popular months for taking vacation for most people
while December has the Christmas holidays where people tend to take time
off to spend with family.

The last section for the exploratory analysis will take a look at
absences by employee type. The Nova Scotia government has both unionized
and non-union employees.

\begin{verbatim}
## Union        3021112
## Non-Union     547798
## Name: employee_type, dtype: int64
\end{verbatim}

Check for statistical differences with union vs non-union employees

\end{document}
